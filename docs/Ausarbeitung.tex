\documentclass[11pt]{article}

\usepackage{blindtext}
\usepackage{titlesec}
\usepackage[hidelinks]{hyperref}

% \subsection{References for the prototype}
% 
% Wenn wir irgendwas referenzieren wollen, vor allem andere Algorithmen. Sollten wir die Quelle in Quellen.bib eintragen und mit \cite{Quelle} zitieren
% Das System beruht auf Bibtex, das muss also runtergeladen werden
% 
% \cite{berger_rompel}

\title{Analyse und Verbesserung bestehender Algorithmen zur Lösung des Graphfärbeproblems}
\author{Menschen} % TODO

\renewcommand*\contentsname{Inhaltsverzeichnis} % Hier kann der Name des Inhaltsverzeichnisses verändert werden

\begin{document}
\maketitle

\tableofcontents

\section{Kurzfassung} % Was haben wir gemacht, ein Absatz
\section{Begriffsklärung} % TODO, laufend aktualisieren, wenn ein neuer Begriff verwendet wurde
\section{Etablierte Algorithmen}
\subsection{Greedy Algorithmen} % TODO
\subsection{Wigdersons Algorithmen} % TODO
\subsection{Johnsons Algorithmus} % TODO
\subsection{Berger Rompel Algorithmus} % TODO

Der Algorithmus von Berger Rompel
baut auf der Idee von Johnsons und Wigderson auf
und verbessert diese um einige Punkte.
\cite{berger_rompel}

\section{Ergebnisse zu den etablierten Algorithmen} % TODO
\section{Eigens produzierte Algorithmen} % TODO
\section{Ergebnisse zu den eigenen Algorithmen} % TODO
\section{Zusammenfassung und Fazit} % TODO

\bibliographystyle{plain}
\bibliography{Quellen}
\end{document}